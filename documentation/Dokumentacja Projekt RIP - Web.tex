\documentclass{article}
\usepackage{polski}
\usepackage[utf8]{inputenc}
\usepackage{hyperref}

\hypersetup{
    colorlinks=true,
    linkcolor=blue,
    filecolor=magenta,      
    urlcolor=cyan,
    pdfpagemode=FullScreen,
    }

\title{Projekt\textunderscore RIP - Web}
\author{
Szymon Bacański
}


\date{Styczeń 2023}

\usepackage{natbib}
\usepackage{graphicx}

\begin{document}

\maketitle

\section{Opis funkcjonalny systemu}
Celem "Projektu\textunderscore RIP - Web" było utworzenie webowej aplikacji łączącej się z wcześniej utworzonym api. Pobiera ona potrzebne dane z wystawionych endpointów 
\\
Założeniem posiadania aplikacji webowej jest ich popularność. Prawie każda osoba posiada urządzenie umożliwiające korzystanie z przeglądarki web. Od telefonów, przez telewizory nawet po inne urządzenia codziennego użytku.

\section{Streszczenie opisu technologicznego}
Node.js jest środowiskiem pozwalającym na łatwiejsze tworzenie aplikacji webowych korzystających z javascript. Wspiera to najpopularniejsze frameworki jak React, Vue czy Angular. Wspiera ono wiele platform, w tym te najbardziej popularne jak windows czy linux.  
\\
React jest biblioteką języka javascript, wykorzystywana do tworzenia frontendowych aplikacji. Składa się ona z małych, odzielonych od siebie elementów. Pozwalaa ona na utworzenie dynamicznych aplikacji web, które skalują się w zależności od urządzenia.
\\
FullCalendar jest biblioteką javascript, która bez problemu działa z aplikacjami opartymi na najbardziej popularnych bibliotekach jak Ract, Vue czy Angular. Wspomaga ona tworzenie aplikacji, które wykorzystują wyświetlanie danych w stylu kalendarzowym.


\section{Instrukcję lokalnego i zdalnego uruchomienia systemu}
\subsection{Postawienie systemu lokalnie}
Wymagane oprogramowanie:\\
Visual Studio Code lub dowolny inny IDE\\
npm\\
Terminal Windows (nie jest potrzebny osobno jeśli jest wbudowany w IDE)\\
Github Desktop – aby móc wprowadzać zmiany, bądź pobierać aktualizacje jeśli są potrzebne. (nie jest wymagany  jeśli jest wbudowany w IDE, bądź jeśli ktoś posiada zainstalowany pakiet GIT do użytku poprzez terminal)\\\\
Jak postawić środowisko testowe ?\\
\textbf{Jeżeli wszystko zainstalowałeś, przejdź do instrukcji poniżej:}
\begin{enumerate}
    \item Pobranie projektu z repozytorium oraz przejście do jego folderu: \\\\
    \emph{ git clone https://github.com/sbacanski0730/RIP-Rewak-and-PUM-Web.git} \\
   			 \emph{ cd RIP-Rewak-and-PUM-Web} \\
    \item Uruchomienie środowiska deweloperskiego\\\\
    \emph{npm start} \\
    \end{enumerate}
    


\section{Dokumentacja}
Link do dokumentacji na naszych repozytoriach: \\
api: \url{https://github.com/sbacanski0730/RIP-Rewak-and-PUM-API/tree/main/documentation}\\\\
web: \url{https://github.com/sbacanski0730/RIP-Rewak-and-PUM-Web/tree/main/documentation}\\\\
mobile: \url{https://github.com/sbacanski0730/RIP-Rewak-and-PUM-Mobile/tree/main/documentation}\\\\

\section{Wnioski projektowe}

Tworzenie projektu opierającego się na api z podziałem zadań na dwa podzespoły jest wymagające.\\ 
Do pewnego stopnia można obsłużyć dane korzystając z wygenerowanych przykładowych plików json jednak mogą wystąpić różnice między danymi, które końcowo są wyprowadzane z api, a tymi, które znajdowały się w wygenerowanych przykładowych json'ach.\\
Różnice zdań oraz wymagania technologiczne mogą doprowadzić do małych zderzeń między podzespołami, jednak kluczem do napisania działającego projektu jest poradzenie sobie z tym pomimo istniejących problemów.\\
Wiele stresu mogą też powodować zbliżające się terminy, które nieubłaganie wydają się być coraz bliżej. Jednak dobra współpraca i poradzenie sobie z różnicami zdań pozwala wyjść ponad takie problemy i dokończcyć projekt.


\end{document}